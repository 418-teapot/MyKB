\documentclass[tikz]{standalone}

\usepackage{ctex}

\usetikzlibrary{
    arrows,
    arrows.meta,
    calc,
    graphs,
    math,
    positioning,
    quotes,
    shapes,
}

\begin{document}

\begin{tikzpicture}
    \tikzset{
        triangle/.style = {
            draw,
            isosceles triangle,
            isosceles triangle stretches,
            shape border rotate = 90,
            anchor = north,
        }
    }
    \tikzset{
        literal/.style = {
            midway,
            above = 0.5em,
            text centered,
        }
    }

    \def\memory{
        寄存器,
        L1 cache(SRAM),
        L2 cache(SRAM),
        L3 cache(SRAM),
        主存(DRAM),
        本地存储(本地磁盘),
        远程存储(分布式文件系统,网络服务器)
    }

    \foreach \l [count=\i] in \memory {
        \node [triangle, minimum width = \i * 1.8 cm, minimum height = \i * 1.8 cm] (L\i) {};
        \draw (L\i.left corner) -- (L\i.right corner) node[literal] {\l};
        \tikzmath{
            int \j, \i;
            \j = \i - 1;
        }
        \draw (L\i.left corner) node[left = 2em, above = 1.5em] {L\j:};
    }
    \coordinate (mid) at ($(L7.left corner)!1/2!(L7.north)$);
    \coordinate (start) at ($(L7.left corner)!(mid)!(L7.left corner) + (-2em, 0)$);
    \draw [thick, -latex] ($(start) + (0, 1em)$) -- node[left, text width = 5em] {更小、更快、更贵的存储设备} (start |- L7.north);
    \draw [thick, -latex] ($(start) - (0, 1em)$) -- node[left, text width = 5em] {更大、更慢、更便宜的存储设备} (start |- L7.left corner);
\end{tikzpicture}

\end{document}
