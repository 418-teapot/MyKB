\documentclass[tikz]{standalone}

\usepackage{ctex}

\usetikzlibrary{
    calc,
    decorations.pathreplacing,
    math,
    positioning,
    shapes,
}

\begin{document}

\begin{tikzpicture}
    \node [draw, rectangle split, rectangle split parts = 4, align = center,
           rectangle split horizontal = true, text width = 8em] (A)
           {操作系统\nodepart{two}处理器\nodepart{three}主存\nodepart{four}I/O 设备};

    \def\abstractions{
        (A.three split north)/文件,
        (A.two split north)/虚拟内存,
        (A.one split north)/进程,
        (A.north west)/虚拟机
    }

    \foreach \p/\a [count = \i from 0] in \abstractions {
        \tikzmath{
            int \j, \i;
            \j = \i * 25 + 10;
        }
        \draw [decorate, decoration = {brace, raise = \j pt}]
              \p -- node [above = \j pt] {\a} (A.north east);
        \draw [dashed] \p -- ++(0, \j pt);
    }
    \draw [decorate, decoration = {brace, raise = 35pt}]
          (A.one split north) -- node [above = 35pt] {指令集架构} (A.two split north);
    \draw [dashed] (A.north east) -- ++(0, 85pt);

\end{tikzpicture}

\end{document}
