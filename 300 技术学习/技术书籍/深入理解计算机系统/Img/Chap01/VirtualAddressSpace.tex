\documentclass[]{standalone}

\usepackage{color}
\usepackage[usenames, dvipsnames]{xcolor}
\usepackage{tikz}
\usepackage{ctex}

\usetikzlibrary{
    calc,
    decorations.pathreplacing,
    math,
    positioning,
    shapes,
}

\begin{document}

\begin{tikzpicture}
    \node [draw, rectangle split, rectangle split parts = 9, align = center,
           rectangle split empty part height = 1.5em,
           rectangle split part fill = {
               lightgray!70!white,
               lightgray!70!white,
               ProcessBlue!40!white,
               lightgray!70!white,
               ProcessBlue!40!white,
               lightgray!70!white,
               lightgray!70!white,
               lightgray!70!white,
               ProcessBlue!40!white},
           minimum width = 12em] (A)
          {内核虚拟内存
           \nodepart{two}用户栈\\(运行时创建)
           \nodepart{three}
           \nodepart{four}共享库的内存\\映射区域
           \nodepart{five}
           \nodepart{six}运行时堆\\(\texttt{malloc}创建)
           \nodepart{seven}读/写内存
           \nodepart{eight}只读代码与数据
           \nodepart{nine}};
    \draw [-latex] (A.two split) -- ++(0, -0.8em);
    \draw [-latex] (A.three split) -- ++(0, 0.8em);
    \draw [-latex] (A.five split) -- ++(0, 1em);

    \node [left = 0 of A.south west] {0};
    \draw [latex-] (A.eight split west) -- ++(-2em, 0) node [left] {\small{程序起始}};

    \draw [-latex] ($(A.text split east) + (0.5em, 0)$) -- node [right] {\small{用户代码不可见的内存}} ($(A.north east) + (0.5em, 0)$);
    \node [right = 0.5em of A.four east] {\small{\texttt{printf}函数}};
    \draw [decorate, decoration = {brace, raise = 0.5em}]
          (A.six split east) -- node [right = 0.5em] {\small{加载自\texttt{hello}可执行文件}} (A.eight split east);
\end{tikzpicture}

\end{document}
